%%%%%%%%%%%%%%%%%%%%%%%%%%%%%%%%%%%%%%%%%%%%%%%%%%%%%%%%%%%%%%%%%%%%%%%%%%%%%%%%%%%%%%%
%% Latex-Templates, Generic
%% Copyright 2012 Hans-Nikolai Viessmann
%
% This work may be distributed and/or modified under the
% conditions of the LaTeX Project Public License, either version 1.3c
% of this license or (at your option) any later version.
% The latest version of this license is in
%   http://www.latex-project.org/lppl.txt
% and version 1.3c or later is part of all distributions of LaTeX
% version 2005/12/01 or later.
%
% This work has the LPPL maintenance status `maintained'.
% 
% The Current Maintainer of this work is Hans-Nikolai Viessmann.
%
% This work consists of the files main.tex, appendix.tex, assignment.tex, title.tex,
% toc.tex, and listings.tex and uses freely available packages created for TeX/LaTex
% by other individuals
%%%%%%%%%%%%%%%%%%%%%%%%%%%%%%%%%%%%%%%%%%%%%%%%%%%%%%%%%%%%%%%%%%%%%%%%%%%%%%%%%%%%%%%
%%%%%%%%%%%%%%%%%%%%%%%%%%%%%%%%%%%%%%%%%%%%%%%%%%%%%%%%%%%%%%%%%%%%%%%%%%%%%%%%%%%%%%%
% Reset Page Number to 1
%%%%%%%%%%%%%%%%%%%%%%%%%%%%%%%%%%%%%%%%%%%%%%%%%%%%%%%%%%%%%%%%%%%%%%%%%%%%%%%%%%%%%%%
\setcounter{page}{1}
\section{Lab Tasks}
	\subsection{Task 1: Encryption using different ciphers and modes}
		My actions for this task can be summed up by the log of commands I ran, below. I was already quite familiar with OpenSSL so only compared the output from the different block cipher modes, directly on my \href{http://andrewbeveridge.co.uk/comnetsec/}{web server} using \href{http://hexedit.duttke.de}{HexEdit.js}
		\subsubsection{Shell Commands}
			\begin{lstlisting}[label={Shell Commands},caption={Task 1}]
				# pwd
				/home/andrewbe/public_html/comnetsec
				# ls -l
				total 12
				drwxr-xr-x  2 andrewbe andrewbe 4096 Oct 11 13:42 ./
				drwxr-x--- 19 andrewbe nobody   4096 Oct 11 13:41 ../
				-rw-r--r--  1 andrewbe andrewbe   49 Oct 11 13:41 plain.txt
				# openssl enc -aes-128-ecb -e -in plain.txt -out cipher-aes-128-ecb.bin -K 00112233445566778889aabbccddeeff -iv 0102030405060708
				# openssl enc -aes-128-cbc -e -in plain.txt -out cipher-aes-128-cbc.bin -K 00112233445566778889aabbccddeeff -iv 0102030405060708
				# openssl enc -aes-128-cfb -e -in plain.txt -out cipher-aes-128-cfb.bin -K 00112233445566778889aabbccddeeff -iv 0102030405060708
				# openssl enc -aes-128-cfb -e -in plain.txt -out cipher-aes-128-cfb.bin -K 00112233445566778889aabbccddeeff -iv 0102030405060708
				# openssl enc -aes-128-ofb -e -in plain.txt -out cipher-aes-128-ofb.bin -K 00112233445566778889aabbccddeeff -iv 0102030405060708
				# ls -l
				total 28
				drwxr-xr-x  2 andrewbe andrewbe 4096 Oct 11 13:42 ./
				drwxr-x--- 19 andrewbe nobody   4096 Oct 11 13:41 ../
				-rw-r--r--  1 andrewbe andrewbe   64 Oct 11 13:42 cipher-aes-128-cbc.bin
				-rw-r--r--  1 andrewbe andrewbe   49 Oct 11 13:42 cipher-aes-128-cfb.bin
				-rw-r--r--  1 andrewbe andrewbe   64 Oct 11 13:42 cipher-aes-128-ecb.bin
				-rw-r--r--  1 andrewbe andrewbe   49 Oct 11 13:42 cipher-aes-128-ofb.bin
				-rw-r--r--  1 andrewbe andrewbe   49 Oct 11 13:41 plain.txt
				#
			\end{lstlisting}
	\pagebreak
	\subsection{Task 2: Encryption Mode — ECB vs. CBC}
	
		\subsubsection{Shell Commands}
			\begin{lstlisting}[label={Shell Commands},caption={Task 1}]
				# wget http://www.macs.hw.ac.uk/~hwloidl/Courses/F21CN/Labs/CryptoI/pic_original.bmp
				--2013-10-11 14:04:33--  http://www.macs.hw.ac.uk/~hwloidl/Courses/F21CN/Labs/CryptoI/pic_original.bmp
				Resolving www.macs.hw.ac.uk... 137.195.13.48
				Connecting to www.macs.hw.ac.uk|137.195.13.48|:80... connected.
				HTTP request sent, awaiting response... 200 OK
				Length: 184974 (181K) [image/bmp]
				Saving to: `pic_original.bmp'

				100%[=========================================================================>] 184,974     --.-K/s   in 0.002s

				2013-10-11 14:04:34 (99.1 MB/s) - `pic_original.bmp' saved [184974/184974]
				# ls -l
				total 192
				drwxr-xr-x  2 andrewbe andrewbe   4096 Oct 11 14:11 ./
				drwxr-x--- 19 andrewbe nobody     4096 Oct 11 13:41 ../
				-rw-r--r--  1 andrewbe andrewbe 184974 Oct 11 14:04 pic_original.bmp
				# openssl enc -aes-128-ecb -e -in pic_original.bmp -out pic-aes-128-ecb.bmp -K 00112233445566778889aabbccddeeff -iv 0102030405060708
				# openssl enc -aes-128-cbc -e -in pic_original.bmp -out pic-aes-128-cbc.bmp -K 00112233445566778889aabbccddeeff -iv 0102030405060708
				# openssl enc -aes-128-cfb -e -in pic_original.bmp -out pic-aes-128-cfb.bmp -K 00112233445566778889aabbccddeeff -iv 0102030405060708
				# openssl enc -aes-128-ofb -e -in pic_original.bmp -out pic-aes-128-ofb.bmp -K 00112233445566778889aabbccddeeff -iv 0102030405060708
				# openssl enc -aes-128-ofb -e -in pic_original.bmp -out pic-aes-128-ofb.bmp -K 00112233445566778889aabbccddeeff -iv 0102030405060708
				# dd if=pic_original.bmp bs=1 count=54 | tee >(dd conv=notrunc of=pic-aes-128-ecb.bmp) >(dd conv=notrunc of=pic-aes-128-cbc.bmp) >(dd conv=notrunc of=pic-aes-128-cfb.bmp)| dd conv=notrunc of=pic-aes-128-ofb.bmp
				54+0 records in
				54+0 records out
				54 bytes (54 B) copied, 4.7288e-05 s, 1.1 MB/s
				0+1 records in
				0+1 records out
				54 bytes (54 B) copied, 2.1521e-05 s, 2.5 MB/s
				0+1 records in
				0+1 records out
				54 bytes (54 B) copied, 0.000310489 s, 174 kB/s
				0+1 records in
				0+1 records out
				54 bytes (54 B) copied, 0.000288515 s, 187 kB/s
				0+12 records in
				0+1 records out
				54 bytes (54 B) copied, 0.000912583 s, 59.2 kB/s
				#
			\end{lstlisting}
			
			As can be clearly seen from Figure 1 and Figure 2, Electronic codebook (ECB) should really not be used as a cipher as identical plain text blocks are encoded into the the same cipher blocks, meaning any patterns in the original data can be easily identified; it does not provide proper confidentiality of the plain text message.
	\pagebreak
	
	\subsection{Task 3: Encryption Mode — Corrupted Cipher Text}
		
		Prior to executing any commands, I hypothesized about what the results might show. I felt that CBC, OFB and CFB would have a lot of corruption in the decrypted file, as I thought a single byte becoming corrupt would have greater impact on the bytes around it. 
		I felt ECB would be likely to only have one byte corrupt in the decrypted file, at the same offset as the corrupt byte in the encrypted file, due to the results with the image making me feel it encrypted each byte individually.
		
		\subsubsection{Shell Commands}
			\begin{lstlisting}[label={Shell Commands},caption={Task 1}]
				# nano morethan64bytes.txt
				# openssl enc -aes-128-ofb -e -in morethan64bytes.txt -out morethan64bytes-aes-ofb.bin -K 00112233445566778889aabbccddeeff -iv 0102030405060708
				# printf '\x2F' | dd conv=notrunc of=morethan64bytes-aes-ofb.bin bs=1 seek=30
				1+0 records in
				1+0 records out
				1 byte (1 B) copied, 2.1001e-05 s, 47.6 kB/s
				# openssl enc -aes-128-ofb -d -in morethan64bytes-aes-ofb.bin -out morethan64bytes-ofb-corrupt-decrypted.txt -K 00112233445566778889aabbccddeeff -iv 0102030405060708
				# openssl enc -aes-128-cbc -e -in morethan64bytes.txt -out morethan64bytes-aes-cbc.bin -K 00112233445566778889aabbccddeeff -iv 0102030405060708
				# printf '\x2F' | dd conv=notrunc of=morethan64bytes-aes-cbc.bin bs=1 seek=30
				1+0 records in
				1+0 records out
				1 byte (1 B) copied, 1.7753e-05 s, 56.3 kB/s
				# openssl enc -aes-128-cbc -d -in morethan64bytes-aes-cbc.bin -out morethan64bytes-cbc-corrupt-decrypted.txt -K 00112233445566778889aabbccddeeff -iv 0102030405060708
				# openssl enc -aes-128-ecb -e -in morethan64bytes.txt -out morethan64bytes-aes-ecb.bin -K 00112233445566778889aabbccddeeff -iv 0102030405060708
				printf '\x2F' | dd conv=notrunc of=morethan64bytes-aes-ecb.bin bs=1 seek=30
				openssl enc -aes-128-ecb -d -in morethan64bytes-aes-ecb.bin -out morethan64bytes-ecb-corrupt-decrypted.txt -K 00112233445566778889aabbccddeeff -iv 0102030405060708
				# printf '\x2F' | dd conv=notrunc of=morethan64bytes-aes-ecb.bin bs=1 seek=30
				1+0 records in
				1+0 records out
				1 byte (1 B) copied, 1.7091e-05 s, 58.5 kB/s
				# openssl enc -aes-128-ecb -d -in morethan64bytes-aes-ecb.bin -out morethan64bytes-ecb-corrupt-decrypted.txt -K 00112233445566778889aabbccddeeff -iv 0102030405060708
				printf '\x2F' | dd conv=notrunc of=morethan64bytes-aes-cfb.bin bs=1 seek=30
				openssl enc -aes-128-cfb -d -in morethan64bytes-aes-cfb.bin -out morethan64bytes-cfb-corrupt-decrypted.txt -K 00112233445566778889aabbccddeeff -iv 0102030405060708# openssl enc -aes-128-cfb -e -in morethan64bytes.txt -out morethancddeeff -iv 0102030405060708233445566778889aabbc
				# printf '\x2F' | dd conv=notrunc of=morethan64bytes-aes-cfb.bin bs=1 seek=30
				1+0 records in
				1+0 records out
				1 byte (1 B) copied, 1.5508e-05 s, 64.5 kB/s
				# openssl enc -aes-128-cfb -d -in morethan64bytes-aes-cfb.bin -out morethan64bytes-cfb-corrupt-decrypted.txt -K 00112233445566778889aabbccddeeff -iv 0102030405060708
				# ls -la
				total 44
				drwxr-xr-x  2 andrewbe andrewbe 4096 Oct 11 15:11 ./
				drwxr-x--- 19 andrewbe nobody   4096 Oct 11 13:41 ../
				-rw-r--r--  1 andrewbe andrewbe  368 Oct 11 15:11 morethan64bytes-aes-cbc.bin
				-rw-r--r--  1 andrewbe andrewbe  359 Oct 11 15:11 morethan64bytes-aes-cfb.bin
				-rw-r--r--  1 andrewbe andrewbe  368 Oct 11 15:11 morethan64bytes-aes-ecb.bin
				-rw-r--r--  1 andrewbe andrewbe  359 Oct 11 15:09 morethan64bytes-aes-ofb.bin
				-rw-r--r--  1 andrewbe andrewbe  359 Oct 11 15:11 morethan64bytes-cbc-corrupt-decrypted.txt
				-rw-r--r--  1 andrewbe andrewbe  359 Oct 11 15:11 morethan64bytes-cfb-corrupt-decrypted.txt
				-rw-r--r--  1 andrewbe andrewbe  359 Oct 11 15:11 morethan64bytes-ecb-corrupt-decrypted.txt
				-rw-r--r--  1 andrewbe andrewbe  359 Oct 11 15:06 morethan64bytes-ofb-corrupt-decrypted.txt
				-rw-r--r--  1 andrewbe andrewbe  359 Oct 11 15:03 morethan64bytes.txt
				# cat morethan64bytes-cbc-corrupt-decrypted.txt
				The job of waxindÐKw¡òx}åWWdgúently peeves c5intzy kids.
				West quickly gave Bert handsome prizes for six juicy plums.
				Just keep examining every low bid quoted for zinc etchings.
				A quick movement of the enemy will jeopardize six gunboats.
				All questions asked by five watch experts amazed the judge.
				The exodus of jazzy pigeons is craved by squeamish walkers.
				# cat morethan64bytes-cfb-corrupt-decrypted.txt
				TÑÒô«ob of waxing linoleum freÉu±ø:
					 ntzy kids.
				West quickly gave Bert handsome prizes for six juicy plums.
				Just keep examining every low bid quoted for zinc etchings.
				A quick movement of the enemy will jeopardize six gunboats.
				All questions asked by five watch experts amazed the judge.
				The exodus of jazzy pigeons is craved by squeamish walkers.
				# cat morethan64bytes-ecb-corrupt-decrypted.txt
				The job of waxin¡ª¼mBÚ¸a û|Iently peeves chintzy kids.
				West quickly gave Bert handsome prizes for six juicy plums.
				Just keep examining every low bid quoted for zinc etchings.
				A quick movement of the enemy will jeopardize six gunboats.
				All questions asked by five watch experts amazed the judge.
				The exodus of jazzy pigeons is craved by squeamish walkers.
				# cat morethan64bytes-ofb-corrupt-decrypted.txt
				The job of waxing linoleum freguently peeves chintzy kids.
				West quickly gave Bert handsome prizes for six juicy plums.
				Just keep examining every low bid quoted for zinc etchings.
				A quick movement of the enemy will jeopardize six gunboats.
				All questions asked by five watch experts amazed the judge.
				The exodus of jazzy pigeons is craved by squeamish walkers.
				#
			\end{lstlisting}
			
			My original hypothesis was incorrect, ECB, CBC and CFB all have about 13 bytes of corruption in the decrypted output, and OFB only has one corrupt byte, at the same offset. 
			
	\pagebreak
	
	\subsection{Task 4: Checksums}
		
		\subsubsection{Shell Commands}
			\begin{lstlisting}[label={Shell Commands},caption={Task 1}]
				# wget http://www.macs.hw.ac.uk/~hwloidl/Courses/F21CN/Labs/CryptoI/some.aes-128-cbc
				--2013-10-11 15:23:16--  http://www.macs.hw.ac.uk/~hwloidl/Courses/F21CN/Labs/CryptoI/some.aes-128-cbc
				Resolving www.macs.hw.ac.uk... 137.195.13.48
				Connecting to www.macs.hw.ac.uk|137.195.13.48|:80... connected.
				HTTP request sent, awaiting response... 200 OK
				Length: 32 [text/plain]
				Saving to: `some.aes-128-cbc'

				100%[=========================================================================>] 32          --.-K/s   in 0s

				2013-10-11 15:23:16 (6.98 MB/s) - `some.aes-128-cbc' saved [32/32]

				# wget http://www.macs.hw.ac.uk/~hwloidl/Courses/F21CN/Labs/CryptoI/some.txt
				--2013-10-11 15:23:23--  http://www.macs.hw.ac.uk/~hwloidl/Courses/F21CN/Labs/CryptoI/some.txt
				Resolving www.macs.hw.ac.uk... 137.195.13.48
				Connecting to www.macs.hw.ac.uk|137.195.13.48|:80... connected.
				HTTP request sent, awaiting response... 200 OK
				Length: 18 [text/plain]
				Saving to: `some.txt'

				100%[=========================================================================>] 18          --.-K/s   in 0s

				2013-10-11 15:23:24 (3.09 MB/s) - `some.txt' saved [18/18]

				# wget http://www.macs.hw.ac.uk/~hwloidl/Courses/F21CN/Labs/CryptoI/words.txt
				--2013-10-11 15:23:29--  http://www.macs.hw.ac.uk/~hwloidl/Courses/F21CN/Labs/CryptoI/words.txt
				Resolving www.macs.hw.ac.uk... 137.195.13.48
				Connecting to www.macs.hw.ac.uk|137.195.13.48|:80... connected.
				HTTP request sent, awaiting response... 200 OK
				Length: 206662 (202K) [text/plain]
				Saving to: `words.txt'

				100%[=========================================================================>] 206,662     --.-K/s   in 0.04s

				2013-10-11 15:23:29 (4.94 MB/s) - `words.txt' saved [206662/206662]

				# openssl sha1 some.aes-128-cbc
				SHA1(some.aes-128-cbc)= 92ce63d9f3495ca005237eb6cca47302b74c574f
				# openssl sha1 words.txt
				SHA1(words.txt)= b3470280f84575a3db3ec3a6b9df2681ee0f5a18
				# openssl sha1 some.txt
				SHA1(some.txt)= 0b6f3556e8773a3e7c0ed31c634b9fd2a108adcc
				#
			\end{lstlisting}
			
		1) The files have not been tampered with, as the hashes match with those published by the author.
		2) Cryptographic hash functions
		3) This check guarantees file /integrity/, as long as the hashes are checked against hashes which are known to be correct and sent over a secure channel. 

	\pagebreak

	\subsection{Task 5: Known-plaintext attack}
		
	
	\pagebreak
	
	\subsection{Notes}
		\subsubsection{Computing Power}
			My first algorithm for task 5 was terribly inefficient, and took a long time (around 20 minutes) when run from linux01 (Core i7-860 CPU). 
			I decided to try it on bwlf01 instead (Xeon E5504 CPU), as it seemed nobody was using that Beowulf node at the time. It took 12 minutes, a significant improvement, though still not great.
			I then ran it from my own dedicated server (Xeon E5-1650)  and managed to squeeze the runtime down to only 6 minutes.
			At this point I realised the algorithm could be improved and tried another method, but I thought my computing power vs. time findings were worth a mention.
		\subsubsection{OpenSSL}
			I performed all the lab tasks from a virtual machine on my own dedicated server running a similar CentOS version as the MACS lab machines. As such, it had the same openssl-devel package installed as required.
		\subsubsection{Editing Binary Files}
			Some of the tasks in this coursework required me to directly edit and replace parts of binary files. Usually I would do this using a graphical hex editor such as \href{http://hexedit.duttke.de}{HexEdit.js}, but since I was working on a headless server I wanted to find a nicer way to replace portions of a file with another. 
			After a little trial and error (and a brief reminder of the skip/seek flags) I ended up using dd, and found it was the perfect tool for the job, when you know your offsets as we did for task 2.
			This way, I avoided the awkward interface of shed, Emacs or (ugh) Vim for binary replacement!  
			
	\pagebreak
			
	\begin{figure}  
	\centering
	\includegraphics{pic_original}
	\caption{Original Image}
	
	\includegraphics{pic-aes-128-ecb}
	\caption{AES-ECB Encrypted Image}
	
	\includegraphics{pic-aes-128-cbc}
	\caption{AES-CBC Encrypted Image}
	
	\includegraphics{pic-aes-128-ofb}
	\caption{AES-OFB Encrypted Image}
	\end{figure}
			
			